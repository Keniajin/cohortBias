A cohort study is a longitudinal observational study in which a study population
(i.e.\ a cohort)  is selected and followed up in
time~\citep{DosSantosSilva1999,Rothman2012}.
Members of the cohort share a common experience (e.g. Kennel Club registered
Labrador Retrievers born after January \nth{1}, 2010; \citealp{Clements_2013}) or
condition (e.g.\ litters from \emph{A.\ pleuropneumoniae} infected sows; \citealp{Tobias_2014}).
Two cohorts are often included in these longitudinal studies, one experiencing a
putative causal event or condition (exposed cohort), and the other being an
unexposed (reference) cohort.
Cohort study is the standard study design to estimate the incidence of diseases
and identify their natural history, by analyzing the association between a
baseline exposure and risk of disease over the follow-up period.
This type of study is characterize by the identification of a disease-free
population, i.e.\ subjects with the outcome at baseline are excluded from the
follow-up, and their exposure to a risk factor is assessed.
The frequency of the outcome (generally the incidence of a disease or death) is
measured and related to exposure status, expressed as a risk ratio (RR).
Therefore it is assumed that prevalent and non-prevalent cases can be
differentiated with no error so that only susceptible individuals are included
in the follow-up.
Incident cases are likewise supposed to be correctly identified.

However, any measurement is prone to potential errors, as a result of subjective
evaluations, imperfect diagnostic tests, reporting errors (deliberate or not),
recall deficiencies, or clerical errors.
Obtaining ``error-free'' measurements is a desirable objective but it is
usually much more expensive to use ``gold-standard'' measurements, or they
are simply not available, leaving the researcher with ``less-than-ideal''
measurement tools.
Such errors of measurement or misclassification in exposure variables, outcomes
or confounders can bias inferences drawn from the data collected, often
substantially~\citep{Quade1980}, or decrease the power of the
study~\citep{Bross1954,WHITE_1986}.
Many studies have looked into the effect of misclassification on statistical
inferences.
These evaluations were either within a cross-sectional study framework,
assessing biased prevalence, or for cohort study designs, evaluating biased
incidence rate or RR estimates based on misclassification at one of the two
time-points (initial assessment or follow-up).
However, both observations at risk and incident cases can be wrongly identified
in longitudinal studies, leading to selection and misclassification biases,
respectively.

The objective of this paper was to evaluate the relative impact of selection and
misclassification biases resulting from misclassification, together, on measures
of incidence and RR.

%%% Local Variables:
%%% ispell-local-dictionary: "canadian"
%%% mode: latex
%%% eval: (flyspell-mode 1)
%%% TeX-master: t
%%% reftex-default-bibliography: ("./bias.bib")
%%% End:
