To investigate the impact of concomitant selection and misclassification biases
on measure of disease frequency, data sets from a hypothetical cohort study with
two samples collected one time unit apart were simulated and analyzed based on
specific test and disease characteristics, for a stable population over the
follow-up time, and with no elimination of disease or clustering of
observations.
Direction and magnitude of bias due to selection, misclassification, and total
bias was assessed for diagnostic test sensitivity (Se) and specificity (Sp)
ranging from 0.7 to 1.0 (0.7, 0.75, 0.8, 0.85, 0.9, 0.95, 0.98, 0.99, 1) and 0.8
to 1.0 (0.8, 0.85, 0.9, 0.95, 0.98, 0.99, 1), respectively, and for specific
disease contexts, i.e.\ disease prevalences of 5 and 20\%, and disease
incidences of 0.01, 0.05, and 0.1 cases/animal-time unit.
The true case status (\(S_1\)) on first sample collection was used to identify
observations at risk at the beginning of the cohort, while the second (\(S_2\))
was used to identify the true outcome.
A hypothetical exposure with known strength of association (RR
\raise.17ex\hbox{$\scriptstyle\sim$}\num{3.0}) was also generated.
For demonstration purpose, simulations were also ran with a weaker RR of
\raise.17ex\hbox{$\scriptstyle\sim$}\num{1.5}.
A total of \numprint{1000} cohort studies of \numprint{1000} observations each
were simulated for these six disease contexts where the same diagnostic test was
used to identify observations at risk at beginning of the cohort and incident
cases at its end.
On each datasets new \(S_{1}'\) and \(S_{2}'\) variables were generated by
applying the scenario misclassification parameters to the \(S_1\) and \(S_2\)
samples.
Incidence and measures of association with the hypothetical exposure were then
computed using first the \(S_{1}'\) and \(S_{2}'\) variables (total bias), then
\(S_{1}'\) and \(S_2\) (selection bias only), and finally the \(S_1\) and
\(S_{2}'\) variables (misclassification bias only).

Disease incidence was computed as the number of new cases at the end of the
cohort divided by the number at risk at its beginning.
Risk ratio was computed as the  ratio of the risk of disease among observations
who were exposed to the risk factor, to the risk among observations who were
unexposed~\citep{Rothman2012}.
Data sets generation and estimation procedures were realized in
\R~\citep{Rsystem}, and simulation code is available at
\url{https://github.com/dhaine/cohortBias}.

%%% Local Variables:
%%% ispell-local-dictionary: "canadian"
%%% mode: latex
%%% eval: (flyspell-mode 1)
%%% TeX-master: t
%%% reftex-default-bibliography: ("./bias.bib")
%%% End:
