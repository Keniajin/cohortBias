%%%%%%%%%%%%%%%%%%%%%%%%%%%%%%%%%%%%%%%%%%%%%%%%%%%%%%%%%%%%%%%%%%%%%%%%%%%%%%%%%%%%%%%%%%%%%%%%%%%%%%%%%%%%%%%%%%%%%%%%%%%%%%%%%%%%%%%%%%%%%%%%%%%%%%%%%%%
% This is just an example/guide for you to refer to when producing your supplementary material for your Frontiers article.                                 %
%%%%%%%%%%%%%%%%%%%%%%%%%%%%%%%%%%%%%%%%%%%%%%%%%%%%%%%%%%%%%%%%%%%%%%%%%%%%%%%%%%%%%%%%%%%%%%%%%%%%%%%%%%%%%%%%%%%%%%%%%%%%%%%%%%%%%%%%%%%%%%%%%%%%%%%%%%%

%%% Version 2.4 Generated 2017/05/01 %%%
%%% You will need to have the following packages installed: datetime, fmtcount, etoolbox, fcprefix, which are normally inlcuded in WinEdt. %%%
%%% In http://www.ctan.org/ you can find the packages and how to install them, if necessary. %%%
%%%  NB logo1.jpg is required in the path in order to correctly compile front page header %%%

\documentclass[utf8]{frontiers_suppmat} % for all articles
\usepackage{url,hyperref,lineno,microtype,subcaption}
\usepackage[onehalfspacing]{setspace}

\graphicspath{{graphics/pdf/}}

% Leave a blank line between paragraphs instead of using \\

\begin{document}
\onecolumn
\firstpage{1}

\title[Supplementary Material]{{\helveticaitalic{Supplementary Material}}:
\\ \helvetica{Selection and Misclassification Biases in
  Longitudinal Studies}} %Please insert the title of your article here


\maketitle


% \section{Supplementary Data}

% Supplementary Material should be uploaded separately on submission. Please include any supplementary data, figures and/or tables. 

% Supplementary material is not typeset so please ensure that all information is clearly presented, the appropriate caption is included in the file and not in the manuscript, and that the style conforms to the rest of the article. 

\section{Supplementary Figures}

% For more information on Supplementary Material and for details on the different file types accepted, please see  \href{http://home.frontiersin.org/about/author-guidelines#SupplementaryMaterial}{the Supplementary Material section}  of the Author Guidelines.


% \subsection{Figures}

%%% There is no need for adding the file termination, as long as you indicate where the file is saved. In the examples below the files (logo1.jpg and logos.jpg) are in the Frontiers LaTeX folder
%%% If using *.tif files convert them to .jpg or .png
%%%  NB logo1.jpg is required in the path in order to correctly compile front page header %%%

\begin{figure}[htbp]
  \begin{center}
    \includegraphics[scale=.95]{master-incidence5_contourX-1}
    \end{center}
  \caption{Estimated incidence rate as a function of test sensitivity and
    specificity, a disease prevalence of 5\%, and true disease incidence (0.01,
    0.05, 0.1 case/animal-time unit) when using an imperfect test at baseline
    (selection bias) or at follow-up (misclassification bias). True incidence
    rate is found at the upper right corner (i.e.\ perfect sensitivity and
    specificity).}
  \label{fig:incidence_contourX5}
\end{figure}

\begin{figure}[htbp]
  \begin{center}
    \includegraphics[scale=.95]{master-incidence20_contourX-1}
    \end{center}
  \caption{Estimated incidence rate as a function of test sensitivity and
    specificity, a disease prevalence of 20\%, and true disease incidence (0.01,
    0.05, 0.1 case/animal-time unit) when using an imperfect test at baseline
    (selection bias) or at follow-up (misclassification bias). True incidence
    rate is found at the upper right corner (i.e.\ perfect sensitivity and
    specificity).}
  \label{fig:incidence_contourX20}
\end{figure}

\begin{figure}[htbp]
  \begin{center}
    \includegraphics[scale=.95]{master-risk5_contourX-1}
    \end{center}
  \caption{Estimated risk ratio as a function of test sensitivity and
    specificity, a disease prevalence of 5\%, and true disease incidence (0.01,
    0.05, 0.1 case/animal-time unit) for an exposure with a true measure of
    association corresponding to a risk ratio of \(3.0\) when using an imperfect
    test at baseline (selection bias) or at follow-up (misclassification bias).
    True risk ratio is found at the upper right corner (i.e.\ perfect
    sensitivity and specificity).}
  \label{fig:risk_contourX5}
\end{figure}

\begin{figure}[htbp]
  \begin{center}
    \includegraphics[scale=.95]{master-risk20_contourX-1}
    \end{center}
  \caption{Estimated risk ratio as a function of test sensitivity and
    specificity, a disease prevalence of 20\%, and true disease incidence (0.01,
    0.05, 0.1 case/animal-time unit) for an exposure with a true measure of
    association corresponding to a risk ratio of \(3.0\) when using an imperfect
    test at baseline (selection bias) or at follow-up (misclassification bias).
    True risk ratio is found at the upper right corner (i.e.\ perfect
    sensitivity and specificity).}
  \label{fig:risk_contourX20}
\end{figure}

\begin{figure}[htbp]
  \begin{center}
    \includegraphics[scale=.95]{master-risk15_contour-1}
  \end{center}
  \caption{Estimated risk ratio as a function of test sensitivity and
    specificity, disease prevalence (5 or 20\%), and true disease incidence
    (0.01, 0.05, 0.1 case/animal-time unit) for an exposure with a true measure
    of association corresponding to a risk ratio of \(1.5\) when using an
    imperfect test both at baseline and follow-up (i.e.\ total bias). True risk
    ratio is found at the upper right corner (i.e.\ perfect sensitivity and
    specificity).}
  \label{fig:incidence_risk15}
\end{figure}

\begin{figure}[htbp]
  \begin{center}
    \includegraphics[scale=.95]{master-RR15_Se_PrInc-1}
  \end{center}
  \caption{Estimated risk ratio as a function of test specificity and disease
    risk, and for a sensitivity of 95\%, when using an imperfect test both at
    baseline and follow-up. True risk ratio = \(1.5\).}
  \label{fig:apparent_RR15}
\end{figure}

\begin{figure}[htbp]
  \begin{center}
    \includegraphics[scale=.95]{master-risk15_5_contourX-1}
  \end{center}
  \caption{Estimated risk ratio as a function of test sensitivity and
    specificity, a disease prevalence of 5\%, and true disease incidence (0.01,
    0.05, 0.1 case/animal-time unit) for an exposure with a true measure of
    association corresponding to a risk ratio of \(1.5\) when using an imperfect
    test at baseline (selection bias) or at follow-up (misclassification bias).
    True risk ratio is found at the upper right corner (i.e.\ perfect
    sensitivity and specificity).}
  \label{fig:risk15_contourX5}
\end{figure}

\begin{figure}[htbp]
  \begin{center}
    \includegraphics[scale=.95]{master-risk15_20_contourX-1}
  \end{center}
  \caption{Estimated risk ratio as a function of test sensitivity and
    specificity, a disease prevalence of 20\%, and true disease incidence (0.01,
    0.05, 0.1 case/animal-time unit) for an exposure with a true measure of
    association corresponding to a risk ratio of \(1.5\) when using an imperfect
    test at baseline (selection bias) or at follow-up (misclassification bias).
    True risk ratio is found at the upper right corner (i.e.\ perfect
    sensitivity and specificity).}
  \label{fig:risk15_contourX20}
\end{figure}

%%% If you are submitting a figure with subfigures please combine these into one image file with part labels integrated.
%%% If you don't add the figures in the LaTeX files, please upload them when submitting the article.
%%% Frontiers will add the figures at the end of the provisional pdf automatically
%%% The use of LaTeX coding to draw Diagrams/Figures/Structures should be avoided. They should be external callouts including graphics.


%\bibliographystyle{frontiersinSCNS_ENG_HUMS} %  for Science, Engineering and Humanities and Social Sciences articles, for Humanities and Social Sciences articles please include page numbers in the in-text citations
%\bibliographystyle{frontiersinHLTH&FPHY} % for Health and Physics articles
%\bibliography{test}

\end{document}
