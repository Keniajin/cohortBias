Total biases resulting from selection and misclassification errors and according
to given disease prevalence, Se, and Sp are illustrated for disease incidence
and RR in \Cref{fig:incidence_contour,fig:incidence_risk}, respectively.
Imperfect sensitivity to identify individuals at risk and imperfect specificity
to identify incident cases lead to a mild under-estimation of the observed
disease incidence (Figures~S1 and S2 in Supplementary Material).
But the combination of the two biases, at baseline and follow-up, revealed the
importance of a good to excellent specificity over sensitivity for the
diagnostic test.
Small divergence from perfect specificity extended quickly to disease incidence
over-estimation as true prevalence increased and true incidence decreased
(\Cref{fig:apparent_incidence_5,fig:apparent_incidence_20,fig:apparent_RR}).
Selection and misclassification biases of a low prevalent and incident disease,
diagnosed with close to perfect specificity, were minimal, reflecting the
importance of choosing a highly specific test to improve unit at risk and case
identification.
The same effect was also observed with RR estimations (Figures~S3 and S4).

%%% Local Variables:
%%% ispell-local-dictionary: "canadian"
%%% mode: latex
%%% TeX-master: t
%%% reftex-default-bibliography: ("./bias.bib")
%%% End:
