Total biases resulting from selection and misclassification errors and according
to given disease prevalence, Se, and Sp are illustrated for disease incidence
and RR in \Cref{fig:incidence_contour,fig:incidence_risk}, respectively.
These figures are contour plots where the lines are curves in the \(x, y\)-plane
along which the function of the two variables on the vertical and horizontal
axes (i.e.\ Se and Sp) has a constant value, i.e.\ a curve joins points of equal
value~\citep{courant_1996}.
The true incidence rate (or RR) is therefore to be found at the upper right
corner of the plot.
For example, in the bottom left panel of \Cref{fig:incidence_contour} the second
line from the bottom is labelled 0.22.
This line shows that, for a 5\% disease prevalence and a true incidence rate of
0.1 case/animal-time unit, an apparent incidence estimate of 0.22 will be
achieved by any combination of Sp and Se on this line (e.g.\ Sp = 0.845, Se =
0.7 or Sp = 0.87 and Se = 1.00). 
Imperfect sensitivity to identify individuals at risk at baseline and imperfect
specificity to identify incident cases led to a mild under-estimation of the
observed disease incidence (Figures~S1 and S2 in Supplementary Material).
From these graphs we could also note that Sp does not influence selection bias
while Se does not matter for misclassification bias.
Of the two, misclassification bias had a much bigger effect than selection bias.
But overall, the combination of the two biases, at baseline and follow-up,
revealed the importance of a good to excellent specificity relative to
sensitivity for the diagnostic test.
Small divergence from perfect specificity extended quickly to disease incidence
over-estimation as true prevalence increased and true incidence decreased
(\Cref{fig:apparent_incidence_5,fig:apparent_incidence_20,fig:apparent_RR}).
Selection and misclassification biases of a low prevalent and incident disease,
diagnosed with close to perfect specificity, were minimal, reflecting the
importance of choosing a highly specific test to improve identification of
animal (or individual) unit at risk and incident case identification.
The same effect was also observed with RR estimations (Figures~S3 and S4).

%%% Local Variables:
%%% ispell-local-dictionary: "canadian"
%%% mode: latex
%%% TeX-master: t
%%% reftex-default-bibliography: ("./bias.bib")
%%% End:
