Using imperfect tests may lead to biased estimates of disease frequency and
measures of association.
Many studies have looked into the effect of misclassification on statistical
inferences.
These evaluations were either within a cross-sectional study framework,
assessing biased prevalence, or for cohort study designs, evaluating biased
incidence rate or risk ratio estimates based on misclassification at one
of the two time-points (initial assessment or follow-up).
However, both observations at risk and incident cases can be wrongly
identified in longitudinal studies, leading to selection and misclassification
biases, respectively.
The objective of this paper was to evaluate the relative impact of selection
and misclassification biases resulting from misclassification, together, on
measures of incidence and risk ratio.

To investigate impact on measure of disease frequency, data sets from a
hypothetical cohort study with two samples collected one month apart were
simulated and analyzed based on specific test and disease characteristics, with
no elimination of disease during the sampling interval or clustering of
observations.
Direction and magnitude of bias due to selection, misclassification, and total
bias was assessed for diagnostic test sensitivity and specificity ranging from
0.7 to 1.0 and 0.8 to 1.0, respectively, and for specific disease contexts,
i.e.\ disease prevalences of 5 and 20\%, and disease incidences of 0.01, 0.05,
and 0.1 cases/animal-month.
A hypothetical exposure with known strength of association
% (RR \raise.17ex\hbox{$\scriptstyle\sim$}\num{3.0})
was also generated.
A total of \numprint{1000} cohort studies of \numprint{1000} observations each
were simulated for these six disease contexts where the same diagnostic test
was used to identify observations at risk at beginning of the cohort and
incident cases at its end.

Our results indicated that the departure of the estimates of disease incidence
and risk ratio from their true value were mainly a function of test
specificity, and disease prevalence and incidence.
% Imperfect sensitivity to identify individuals at risk and imperfect
% specificity to identify incident cases lead to a mild under-estimation of the
% observed disease incidence.
The combination of the two biases, at baseline and follow-up, revealed the
importance of a good to excellent specificity relative to sensitivity for the
diagnostic test.
Small divergence from perfect specificity extended quickly to disease
incidence over-estimation as true prevalence increased and true incidence
decreased.
% Selection and misclassification biases of a low prevalent and incident
% disease, diagnosed with close to perfect specificity, were minimal, reflecting
% the importance of choosing a highly specific test to improve unit at risk and
% case identification.
A highly sensitive test to exclude diseased subjects at baseline was of less
importance to minimize bias than using a highly specific one at baseline.
% For instance, when a test with high specificity cannot be used, one could use
% a less performant test twice at recruitment or for identifying incident cases
% and with a serial interpretation.
% The lost in sensitivity of such an approach would caused little bias, compared
% to the potential gains due to the increased specificity.

Near perfect diagnostic test attributes were even more important to obtain a
measure of association close to the true risk ratio, according to specific
disease characteristics, especially its prevalence.
Low prevalent and high incident disease lead to minimal bias if disease is
diagnosed with high sensitivity and close to perfect specificity at baseline and
follow-up.
For more prevalent diseases we observed large risk ratio biases towards the
null value, even with near perfect diagnosis.
% This bias also got larger as incidence decreased.
% For diseases with moderate to high prevalence (20\%), the biases can be so
% important that a study using a test with a sensitivity or specificity < 0.95
% would have very little power to identify any measure of association with
% exposures.
% Even with prevalence of disease of 5\%, a dramatic loss of power is to be
% expected when imperfect tests are used.
  
%%% Local Variables:
%%% ispell-local-dictionary: "canadian"
%%% mode: latex
%%% eval: (flyspell-mode 1)
%%% TeX-master: t
%%% reftex-default-bibliography: ("./bias.bib")
%%% End:
