Our results indicated that the departure of the estimates of disease incidence
and risk ratio from their true value were mainly a function of test specificity,
and disease prevalence and incidence.
Imperfect sensitivity to identify individuals at risk and imperfect
specificity to identify incident cases led to a mild under-estimation of the
observed disease incidence.
The combination of the two biases, at baseline and follow-up, revealed the
importance of a good to excellent specificity over sensitivity for the
diagnostic test.
Small divergence from perfect specificity extended quickly to disease
incidence over-estimation as true prevalence increased and true incidence
decreased.
Selection and misclassification biases of a low prevalent and incident
disease, diagnosed with close to perfect specificity, were minimal, reflecting
the importance of choosing a highly specific test to improve unit at risk and
case identification.
A highly sensitive test to exclude diseased subjects at baseline was of less
importance to minimize bias than using a highly specific one at this time point.
Of course, the situation would be different in a population with a very high
disease prevalence.
For most diseases, however, the tendency is to have a large proportion of
healthy animals and a small proportion of diseased ones.
The range of diseases prevalence investigated in our study (5--20\%) would
therefore cover most disease scenarios seen in veterinary, and perhaps, human
studies.

Near perfect diagnostic test attributes were even more important to obtain a
measure of association close to the true risk ratio, according to specific
disease characteristics, especially its prevalence.
Low prevalent and high incident disease led to minimal bias if disease was
diagnosed with high sensitivity and close to perfect specificity.
For more prevalent diseases we observed large risk ratio biases towards the
null value, even with near perfect diagnosis.
This bias also got larger as incidence decreased.
For diseases with moderate to high prevalence (20\%), the biases could be so
important that a study using a test with a sensitivity or specificity < 0.95
would have very little power to identify any measure of association with
exposures.
Even with prevalence of disease of 5\%, a dramatic loss of power is to be
expected when imperfect tests are used.

It is already recognized that misclassification of outcome or exposure during
follow-up leads to bias toward null in the estimated
associations~\citep{Bross1954,Copeland1977,FLEGAL_1986} as well as reduced
statistical efficiency by loss of power~\citep{WHITE_1986} and confidence
intervals of the parameters estimates that are too narrow~\citep{Neuhaus_1999}.
However this bias toward the null value is strictly true only when
misclassification is the same in the two compared groups, i.e.\ exposure and
covariates status do not influence Se and/or
Sp~\citep{Copeland1977,Sorahan_1994,Neuhaus_1999}.
In this case, we have non-differential misclassification.
As shown previously by \cite{Copeland1977}, misclassification bias depends
primarily on the Sp of the test used and increase with disease rarity, with most
of the bias occurring even before the specificity drops below 85\%.
With Se and Sp as high as 0.90 and 0.96, respectively, RR is already
substantially biased (1.5 instead of 2)~\citep{Copeland1977}, but when Sp is
perfect, bias is absent~\citep{Poole1985}.
When disease frequency is low, error in disease diagnosis leads to an increase
in false positives which submerge true positives and dilute measures of
incidence and association.
Bias in RR increases as Se increase and Sp decrease~\citep{WHITE_1986}.
Exposure misclassification alone can cause serious bias on the RR even if Se or
Sp are not lower than 80\%~\citep{Kristensen_1992}.

When misclassification is differential, i.e.\ Se and Sp of outcome
classification is not equal in each true category of exposure (or Se and Sp of
exposure classification is not equal in each true category of outcome),
direction of bias for parameter estimates can be in any
direction~\citep{Dosemeci_1990,Neuhaus_1999,Chen_2013}.
In this case, Se and Sp as low as 90\% can be sufficient to produce high
bias~\citep{Kristensen_1992}.
Direction of the bias can also be in any direction with dependent
misclassification (i.e.\ the errors in one variable are associated with the
errors in an other, \citealp{Assakul_1967,Greenland_1989}), even if
non-differential~\citep{Kristensen_1992}.
The same is found when the exposure variable is not dichotomous but has multiple
levels~\citep{Dosemeci_1990,Weinberg_1994}.
Bias towards the null also requires that selection bias and confounding are
absent~\citep{Jurek_2004}.
There are therefore many situations where bias toward null do not apply.
Even when non-differential misclassification is thought to take place, random
errors in the observed estimates can lead bias away from the
null~\citep{Jurek_2004}.

In cohort studies, non-differential misclassification of disease at baseline,
i.e.\ selection bias, especially imperfect sensitivity, can lead to over- or
under-estimation of the observed RR~\citep{Pekkanen2006}.
This bias can be significant for disease with a low true incidence, a high true
prevalence, a substantial disease duration (i.e.\ as long as the interval between
first and second test), and a poor test Se.
In the presence of misclassification of disease at baseline the observed RR
depend on the association between exposure and disease both at baseline and
during follow-up~\citep{Pekkanen2006}.
Therefore to minimize bias, the standard recommendation is to exclude subjects
with the outcome at baseline from the cohort based on a highly sensitive
test~\citep{Pekkanen2008}.
Then during the follow-up period, case identification should use a highly
specific test having a high positive predictive value~\citep{Brenner1993}.
However \cite{Haine2017} have shown that a more prevalent and incident disease
diagnosed with an imperfect Se and/or Sp will give biased measure of association
despite attempts to improve its diagnosis.

We have shown here that combined misclassification at baseline and follow-up
requires a highly specific test.
If a test with high Sp cannot be used, one could use a less efficient test twice
at recruitment or for identifying incident cases and with a serial
interpretation.
The lost in Se of such an approach would caused little bias, compared to the
potential gains due to the increased Sp.
However, this combined misclassification would also require a highly sensitive
test to estimate an association close to the true RR.
Unfortunately increasing Sp of a test very often decreases its Se, i.e.\ a lower
probability for diseased individuals to be recognized as diseased.
As a results, some classification errors are to be expected leading to biased
parameters estimates.
If classification errors cannot be avoided during the study design stage, the
misclassification bias can be corrected into the analytic stage.
For instance, Se and Sp of the test can be incorporated into the modelling
strategy~\citep{Magder1997}, by performing a probabilistic sensitivity
analysis~\citep{Fox_2005}, or by including the uncertainty in the estimates with
a Bayesian analysis in the form of prior distributions~\citep{McInturff2004}.
A latent class model~\citep{Hui1980} would therefore return the posterior
inference on regression parameters and the Se and Sp of both tests.

Bias in parameters estimates can be important when considering selection and
misclassification biases together in a cohort study.
Our results underscore the need for a careful evaluation of the best available
options to identify at risk and incident cases according to the expected disease
prevalence and incidence of the study. 

%%% Local Variables:
%%% ispell-local-dictionary: "canadian"
%%% mode: latex
%%% eval: (flyspell-mode 1)
%%% TeX-master: t
%%% reftex-default-bibliography: ("./bias.bib")
%%% End:
